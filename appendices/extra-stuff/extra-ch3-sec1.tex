contributions:

<<<terms such as model, neural net, CNN have been used interchangeably throughout this paper>>>

0) The starting off point for our thesis was to run the inference part of the 2020 MPI model (made publicly available by the authors on GitHub) on a videochat frame and find that the generated disparity map was visually inaccurate, whereas it would be comparatively much more visually accurate when run on a cityscape video frame. This could be expected because the 2020 MPI model was trained only on RealEstate10K cityscape video data.   

1) Problem statement 1: How to increase the depth prediction accuracy for videochat frames? 

1) The first step in address problem statement 1 was to recreate the 2020 MPI training procedure. There was no chance of transfer learning because there was only a textual description of the encoder-decoder training network available in the 2020 MPI paper and hence no weights could be procured and the model had to be recreated. 
    0.1) This information was confirmed by the authors dof the 2020 paper due to proprietary code involeved whom we were in touch with and who confirmed the same
    0.2) if the entire training code was availbale, with the weights file, we believe transfer learning would have been the more feasible option and it would have been an interesting data point to compare against. 
    
1) recreating 2020 mpi neural net given that only partial code <<<what parts exactly?>>>> for the same was made available online by the authors 

2) Recreation involved the following:
    2.1) Taking the readily available network information about the various types of encoder-decoder layers involved <<<<etc>>>> from the paper and putting it together with unavailable and recreated aspects of the network like the data loader part, the loss function<<<(s)>>> <<<<<etc>>>
    2.2) Trying to train the network only on the video-chat-relevant Mannequin dataset
    2.3) Fixing the bugs encountered during training
    

3) retrained the recreated model on not just the original RealEstate10K dataset <<<how did the model perform on this original data explained in previous point>>>> but also the Mannequin Challenge dataset, clubbing both datasets together to be fed as input to the recreated MPI model

4) The main reason for extending the input data with the Mannequin Challenge dataset was so the hypothesis of this thesis could be proved <<<<<has the hypothesis 1 been proved?>>>>>

6) Hypothesis a: the recreated model trained on the original RealEstate10K dataset perfoms as well as the 2020 MPI model inference output --- <<<<<<these are the results>>>>>>> 

7) Hypothesis b: Having the model train on just the mannequin dataset would be sufficient to accurately predict the MPI and the disparity map for any input videochat frame or any other video frame involving close-up shots of one or more persons <<<<hypothesis 0 was debunked>>>>

8) Hypothesis c: The recreated MPI model, when originally not being able to predict the disparity map (byproduct of MPI prediction) of close-up shots of people accurately (<<<<<experiments & results section: face being concave when its supposed to be convex>>>>>>), would start to do so by adding a relevant dataset such as the Mannequin Challenge dataset to the original input data.

9) We believe this hypothesis could be true for the following reasons: 
    7.1) The training set has been increased with data relevant to videochat scenario <<<by what percentage???>>>>>>> by the addition of the Mannequin Challenge dataset
    7.2) another main reason for would be due to our meticulous curating of the test set into multiple sub-test-sets characterized by the number of people in the shot, whether half or less than half of the person was captured in the frame, and how close the captured person(s) was to the camera.

10) Finally, we had three different datapoints to compare amongst: 6) 7) 8)



 
/include{figure} training pipeline 
draw.io


inputs and outputs for all components 
dataset text files dowload -----> youtubr downloader downloaded videos -------> COLMAP 3d points ------> 
                                        |
                                        |
                                        |-------------------------------------------------------------->




And the main reason for extending the Mannequin Challenge dataset with the originally used RealEstate10K dataset was so Hypothesis B of this thesis could be proved: ---

\paragraph{Hypothesis A:}
The recreated model trained only on the video-chat-relevant Mannequin Challenge Dataset performs as well as the 2020 MPI model does on real estate data, when it comes to accurately predicting disparity maps of video-chat-relevant frames involving close up shots of one or more persons.

\paragraph{Hypothesis B:}
If hypothesis A is disproved \footnote{Please refer to the Experiments and Results Chapter.}, the recreated MPI model should perform as well as the 2020 MPI model if the Mannequin Challenge training data were extended by the RealEstate10K dataset.

Approach section:

When we ran the inference part of the 2020 Single-View MPI model <<<terms such as model, neural network, CNN have been used interchangeably throughout this thesis>>> on a videochat frame and found that the generated disparity map (a by-product of MPI inference in this network) was visually inaccurate, it became the starting off point for our thesis. Comparatively, the inferred disparity map would be much more visually accurate when run on a real estate video frame. The latter outcome was to be expected because the 2020 Single-View MPI model was exclusively trained on the RealEstate10K video dataset.   

1) Problem statement 1: How to increase the 2020 Single-View MPI model’s depth prediction accuracy for videochat frames or any other image with close up shots of people? 

1) As inference was one of the only parts of the network made publicly available by the authors on GitHub due to the proprietary nature of some other parts of the code, the first step in addressing problem statement 1 was to recreate the 2020 Single-View MPI training procedure. There was no chance of transfer learning because there was only a textual description of this encoder-decoder type training network provided in the 2020 MPI paper. If the entire training code were available, we believe transfer learning would have been either a more feasible option or an interesting data point to compare against. 

2) Recreating and retraining the network involved the following:
    2.1) Taking the readily available network information from the paper --- like details about the various types of encoder-decoder layers involved, etc. --- and putting it in place with other aspects of the network that called for a more involved recreation process like the data loader part and the loss function.
    2.2) Trying to train the network only on the videochat-relevant Mannequin Challenge video dataset which is much smaller than the original RealEstate10K dataset.
    2.3) Fixing the bugs encountered during training.
    2.4) Retraining the recreated model on not just the Mannequin Challenge dataset but also the RealEstate10K dataset by clubbing both datasets together to be used as a singular source of input to the recreated MPI model.
    2.5) This input consisted not only of the video frames extracted by the data loader but also of the corresponding 3D points of the pixels in each frame as triangulated by COLMAP <<<<Please find more information in the Data Section.>>> 

Any input to the model consisted not only of the video frames extracted by the data loader but also of the corresponding 3D points of the pixels in each frame as progressively triangulated with adjacent frames by COLMAP \footnote{Please find more information on preparing the inputs in the Data Section.}
And the main reason for extending the Mannequin Challenge dataset with the originally used RealEstate10K dataset was so the Hypothesis B of this thesis could be proved.


6) Hypothesis A: The recreated model trained only on the video-chat-relevant Mannequin Challenge Dataset performs as well as the 2020 MPI model when it comes to accurately predicting disparity maps of video-chat-relevant frames involving close up shots of one or more persons

7) Hypothesis B: If hypothesis A is disproved <<<<please refer to the experiments and results chapter>>>>, The recreated MPI model would perform as well as the 2020 MPI model if the Mannequin Challenge training data were extended by the RealEstate10K dataset.


9) We believe this hypothesis could be true for the following reasons: 
    7.1) The training set has been increased with data relevant to videochat scenario <<<by what percentage???>>>>>>> by the addition of the Mannequin Challenge dataset
    7.2) another main reason for would be due to our meticulous curating of the test set into multiple sub-test-sets characterized by the number of people in the shot, whether half or less than half of the person was captured in the frame, and how close the captured person(s) was to the camera.

10) Finally, we had three different datapoints to compare amongst: 6) 7) 8)



 
/include{figure} training pipeline 
draw.io


inputs and outputs for all components 
dataset text files dowload -----> youtubr downloader downloaded videos -------> COLMAP 3d points ------> 
                                        |
                                        |
                                        |-------------------------------------------------------------->

contributions:







    0.1) This information was confirmed by the authors dof the 2020 paper due to proprietary code involeved whom we were in touch with and who confirmed the same
   
1) recreating 2020 mpi neural net given that only partial code <<<what parts exactly?>>>> for the same was made available online by the authors 

    2.4) Retraining the recreated model on not just the original RealEstate10K dataset <<<how did the model perform on this original data explained in previous point>>>> but also the Mannequin Challenge dataset, clubbing both datasets together to be fed as input to the recreated MPI model

4) The main reason for extending the input data with the Mannequin Challenge dataset was so the hypothesis of this thesis could be proved <<<<<has the hypothesis 1 been proved?>>>>>

6) Hypothesis A: the recreated model trained on the original RealEstate10K dataset perfoms as well as the 2020 MPI model inference output --- <<<<<<these are the results>>>>>>> 

7) Hypothesis b: Having the model train on just the mannequin dataset would be sufficient to accurately predict the MPI and the disparity map for any input videochat frame or any other video frame involving close up shots of one or more persons <<<<hypothesis 0 was debunked>>>>

8) Hypothesis c: The recreated MPI model, when originally not being able to predict the disparity map (byproduct of MPI prediction) of close-up shots of people accurately (<<<<<experiments & results section: face being concave when its supposed to be convex>>>>>>), would start to do so by adding a relevant dataset such as the Mannequin Challenge dataset to the original input data.


_________________________________________________________________________

\chapter{Implementation}\label{ch:implementation}

\section{Recreating and repurposing ML model from Single-view view synthesis from MPIs}\label{sec:contributions}






These are the steps taken to complete the 2-way/1-way rendering pipeline: -

And these are contributions of this thesis: -
Supplementing the process of generating accurate MPI representations for close up targets like heads and bodies by 
\begin{itemize}
    \item re-training the MPI network with datasets containing stereo images of close up objects  
    \item recreating loss functions from the textual descriptions in the 2020 MPI paper
    \item writing a data loader for MannequinChallenge dataset and RealEstate10k dataset
    \item writing scripts to produce 3D point clouds with COLMAP
    \item adapting the training and testing scripts from the 2020 MPI paper to the aforementioned datasets and loss functions
    \item fixing Nan gradient errors produced by cumprod used in several places in MPI code by replacing it with safe\_cumprod as suggested by on of the authors of the 2020 paper
    \item fixing ValueErrors from our Data Loader
\end{itemize}
This 



These are the steps taken to get both datasets ready for training: ---

Step 0: Download either dataset from the URLs listed in 4 and 5. Both datasets consists of .txt files pertaining to each video that can be downloaded. Each .txt file begins with the downloadable video’s YouTube link on the first line and continues to list the details of ORB-SLAM2 processed frames from the video --- with one line for each frame --- like the timestamp (in microseconds), camera intrinsics, and camera extrinsics .

Step 1: Clone the GitHub repository associated with this thesis from https://github.com/au001/view-synthesis.git
Step 2: Build Dockerfile by running build-run-docker/build-docker.sh. Verify dependency version compatibilities by consulting the build log file if Dockerfile fails to build. Start docker container with build-run-docker/build-docker.sh.


Step 1: Back inside either downloaded dataset directory, run the script \begin{lstlisting}[language=Python] ../view-synthesis/scripts/get_videos.py \end{lstlisting} on train/ folder to download all videos with youtube-dl at 720p resolution. Alternatively run ../view-synthesis/scripts/get_videos_aria2c.py to bolster youtube-dl’s download speed with Aria2 download manager.  

Step 2: Inevitably youtube-dl will throw download errors on the first run as there may be some partial and/or skipped downloads for various reasons ranging from the videos being taken down from YouTube over time to fixable errors intrinsic to youtube-dl. Moreover, although some videos may be completely downloaded, it wouldn’t have been the 720p version that was downloaded due to lack of availability. As of this writing 2663 Mannequin Challenge videos and 67582 RealEstate10K videos were downloadable at 720p resolution and with no download errors on the first attempt. Hence, make sure to also save all the outputs generated by running the previous script to a log file.

Step 3: Extract to a new log file the lines from the previously saved log file that contain specific combinations of the strings “error:” and ": Downloading webpage" Run get_errors.py on get-vid-log-train-202106241546.txt
python3 ../scripts/get_errors.py get-vid-log-train-202106241546.txt train-errors-202106281624.txt
Step 3: Find the names of all the youtube online videos in <train-errors-202106281624.txt> that can be found in this specific pattern
"youtube\] (.*?)\: Download"
python3 ../scripts/get_name_within_pattern.py train-errors-202106281624.txt train-error-names-202106281635.txt
Step 3 can only be executed if there are no .part mp4 downloads remaining in <train/> <test/> or <val/> folders
 
Step 4: Make sure there are no more part downloads
Transfer all part downloads including their respective txt files to a separate folder
find -name '*.part' -or -name '*.aria2' | xargs file > ../train-part-dls.txt
python3 ../scripts/get_name_within_pattern.py train-part-dls.txt train-part-dl-names.txt
python3 ../scripts/transfer_videos.py train-part-dl-names.txt train/ train-attempt2/
Try downloading these videos again in the separate folder
python3 ../scripts/get_videos_aria2c.py train-attempt2/ |& tee get-vid-log-train2-202106282055.txt
It looks like the re-downloaded videos are all still either partial or glitchy -> this proves that it is sufficient to run get_videos.py just once
Step 5: Find the names of all the youtube offline/downloaded videos in <train-error-names-202106290046.txt>
python3 ../scripts/get_err_video_names.py train-error-names-202106290046.txt train/ train-err-vid-names-202106290116.txt
Extraordinarily this step produces tons of duplicates in <train-err-vid-names-202106290116.txt>, so this is how we prune the duplicates
 sort -u train-err-vid-names-202106290116.txt > train-err-vid-names-uniq-202106290116.txt


Step 6: Transfer out all files related to video names in <train-err-vid-names-202106290116.txt>
python3 ../scripts/transfer_videos.py train-err-vid-names-uniq-202106290116.txt train/ train-attempt2/
Step 7: Some good videos are also transferred out inevitably, so let's check the integrity of these videos first
https://stackoverflow.com/questions/34077302/quickly-check-the-integrity-of-video-files-inside-a-directory-with-ffmpeg
Check step 
Step 8: If it seems all mp4 files are replete with integrity, let’s move them out, but the question is how do you also move out the related txt files
https://superuser.com/questions/497434/linux-how-to-find-all-files-with-the-same-name-different-filetype-and-move-them
find . -name '*.mp4' -exec /bin/sh -c 'A=`basename {} .mp4`.txt ; test -f $A && mv {} $A ../train-attempt3/' \;


Step 9: maybe do an inverse find to check if there are any other file types
find ! -name '*.mp4' -and ! -name '*.txt' -and ! -name '*.mp4.part' -and ! -name '*.part.aria2'
Or maybe check all filetypes with file cmd
https://www.geeksforgeeks.org/file-command-in-linux-with-examples/
Step 10: check the integrity of mp4s because some of them are complete like 70 MB mp4s but playback in VLC is black maybe because aria2 only partially downloaded them
find *.mp4 -exec sh -c "ffmpeg -v error -i {} -map 0:1 -f null - 2>{}.log" \;
One log file for each broken mp4
find -name '*.log' -exec rm {} \;
Check if log files are useful


auppulur@elcapitan:/data3/RealEstate10K-data/RealEstate10K/train1-integrity-check-sample.log


Upon random checking we can find that all log files are blank so there are no ffmeg errors to repost of the lack of integrity of any mp4 
Clean up log files
 

Step 11: Some mp4s just don’t have a 720p version period — so we put them aside for future work training in /data3/RealEstate10K-data/RealEstate10K/train-non-720p
Step 12: COLMAP 3d points generation for each frame of each video requires getting the final Dockerfile-mark7 ready
https://docs.opencv.org/master/d7/d9f/tutorial_linux_install.html
http://dlib.net/compile.html
https://github.com/TadasBaltrusaitis/OpenFace/blob/master/docker/Dockerfile
https://github.com/TadasBaltrusaitis/OpenFace/wiki/Unix-Installation
https://github.com/TadasBaltrusaitis/OpenFace/blob/master/install.sh
https://hub.docker.com/layers/geki/colmap/latest/images/sha256-51b967b1f5dc38c1d6d3409453de86ca6386aa4bedfeeda32b621ea703a1092a?context=explore
https://colmap.github.io/install.html
https://github.com/colmap/colmap/blob/dev/docker/Dockerfile
https://demuc.de/colmap/#about
http://ceres-solver.org/installation.html




  Step 12: fix cuda issue and make Dockerfile access GPUs
-- Found CUDA: /usr/local/cuda (found suitable version "10.1", minimum required is "7.5") 
-- Found CUDA, but CMake was unable to find the cuBLAS libraries that should be part of every basic CUDA install. Your CUDA install is somehow broken or incomplete. Since cuBLAS is required for dlib to use CUDA we won't use CUDA.
-- DID NOT FIND CUDA
-- Disabling CUDA support for dlib.  DLIB WILL NOT USE CUDA
tf-docker /home/auppulur > uname -r
3.10.0-1127.18.2.el7.x86_64
Anyways let's drop this now after 2 days and get back to the prize




  
Step 13: We may try to make our own Dockerfile better by looking at the original base tensorflow docker hub github one or even nvidia/cuda ones 
https://github.com/tensorflow/tensorflow/blob/master/tensorflow/tools/dockerfiles/dockerfiles/gpu.Dockerfile
https://gitlab.com/nvidia/container-images/cuda/blob/master/dist/11.4.1/ubuntu18.04/devel/Dockerfile  
Finally mpi:mark9 works as I’m inclined to see it work just because of 
FROM nvidia/cuda:10.2-devel-ubuntu18.04
Now let’s increase version numbers starting with a compatibility check:
https://www.tensorflow.org/install/source
  tensorflow-2.4.0	 Python 3.6-3.8	GCC 7.3.1	Bazel 3.1.0	  CuDNN 8.0 	11.0
I have no name!@38dc6ee5b995:~/mpi$ uname -m && cat /etc/*release                                                               
x86_64
DISTRIB_ID=Ubuntu
DISTRIB_RELEASE=18.04
DISTRIB_CODENAME=bionic
DISTRIB_DESCRIPTION="Ubuntu 18.04.5 LTS"
NAME="Ubuntu"
VERSION="18.04.5 LTS (Bionic Beaver)"
ID=ubuntu
ID_LIKE=debian
PRETTY_NAME="Ubuntu 18.04.5 LTS"
VERSION_ID="18.04"
HOME_URL="https://www.ubuntu.com/"
SUPPORT_URL="https://help.ubuntu.com/"
BUG_REPORT_URL="https://bugs.launchpad.net/ubuntu/"
PRIVACY_POLICY_URL="https://www.ubuntu.com/legal/terms-and-policies/privacy-policy"
VERSION_CODENAME=bionic
UBUNTU_CODENAME=bionic


  

Step 14: Potentially the end game for Dockerfile
https://www.quantstart.com/articles/installing-tensorflow-22-on-ubuntu-1804-with-an-nvidia-gpu/
But mainly this: /data/auppulur/helper-projects/openpose-epic/Dockerfile-cylin 
Actually this for cudnn and tensorrt: https://www.tensorflow.org/install/gpu 
Step 4/21 : RUN uname -r
 ---> Running in c25427441d20
3.10.0-1127.18.2.el7.x86_64
Gotta fix this someday: 
debconf: delaying package configuration, since apt-utils is not installed
No need NOTE if you install apt-utils, you will get other warnings (because now the installer can run interactive config and will attempt that
https://stackoverflow.com/questions/51023312/docker-having-issues-installing-apt-utils
Take AWAY: Even the positions of apt-get packages make all the difference
  

Step 15: OpenFace does not take anything other than opencv-4.1.0 and dlib-19.??
Step 16: list all compiler versions including gcc
I have no name!@2a72b8e508ee:/home/auppulur$ dpkg --list | grep compiler
ii  cuda-compiler-10-2                     10.2.89-1                           amd64        CUDA compiler
ii  g++                                    4:7.4.0-1ubuntu2.3                  amd64        GNU C++ compiler
ii  g++-7                                  7.5.0-3ubuntu1~18.04                amd64        GNU C++ compiler
ii  g++-8                                  8.4.0-1ubuntu1~18.04                amd64        GNU C++ compiler
ii  gcc                                    4:7.4.0-1ubuntu2.3                  amd64        GNU C compiler
ii  gcc-7                                  7.5.0-3ubuntu1~18.04                amd64        GNU C compiler
ii  gcc-8                                  8.4.0-1ubuntu1~18.04                amd64        GNU C compiler
ii  libllvm10:amd64                        1:10.0.0-4ubuntu1~18.04.2           amd64        Modular compiler and toolchain technologies, runtime library
ii  libxkbcommon0:amd64                    0.8.2-1~ubuntu18.04.1               amd64        library interface to the XKB compiler - shared library
  

Step 17: Set required python version to be top priority
https://medium.com/analytics-vidhya/how-to-install-and-switch-between-different-python-versions-in-ubuntu-16-04-dc1726796b9b
But mainly this:
https://stackoverflow.com/questions/54633657/how-to-install-pip-for-python-3-7-on-ubuntu-18
Step 18: Finally in mpi:mark12 I attained 
Step 9/26 : RUN which python3 && which python &&     python3 --version && python --version
 ---> Running in ddaae1fe45d7
/usr/bin/python3
/usr/bin/python
Python 3.7.11
[91mPython 2.7.17
Step 19: diff file1 file2


Step 20: mpi-mark13-build-log-202108160139.txt
CMake Error: The following variables are used in this project, but they are set to NOTFOUND.
Please set them or make sure they are set and tested correctly in the CMake files:
CUDA_curand_LIBRARY (ADVANCED)
    linked by target "dlib" in directory /dlib/dlib
CUDA_cusolver_LIBRARY (ADVANCED)
    linked by target "dlib" in directory /dlib/dlib
Perhaps it’s time to reinstall seemingly broken CUDA with https://developer.nvidia.com/cuda-downloads?target_os=Linux&target_arch=x86_64&Distribution=Ubuntu&target_version=18.04&target_type=deb_local and https://www.tensorflow.org/install/gpu


Step 21: two nvidia drivers now 450 and 470
ls -ltar /usr/lib/x86_64-linux-gnu/libnvidia-ptx*
-rwxr-xr-x. 1 root root  9947144 Aug  5  2020 /usr/lib/x86_64-linux-gnu/libnvidia-ptxjitcompiler.so.450.51.06
-rw-r--r--. 1 root root 11144376 Jul 13 16:12 /usr/lib/x86_64-linux-gnu/libnvidia-ptxjitcompiler.so.470.57.02
lrwxrwxrwx. 1 root root       37 Jul 13 21:37 /usr/lib/x86_64-linux-gnu/libnvidia-ptxjitcompiler.so.1 -> libnvidia-ptxjitcompiler.so.470.57.02lrwxrwxrwx. 1 root root       29 Jul 13 21:37 /usr/lib/x86_64-linux-gnu/libnvidia-ptxjitcompiler.so -> libnvidia-ptxjitcompiler.so.1
cat /proc/driver/nvidia/version


Step 22: Proof that I need to manually reinstall CUDA even in a supposedly CUDA enabled image
mpi:mark18 where I just installed cudnn and tensorrt alone but still no packages could locate cublas and other CUDA stuff
-- Found CUDA: /usr/local/cuda (found suitable version "10.1", minimum required is "7.5") 
-- Found CUDA, but CMake was unable to find the cuBLAS libraries that should be part of every basic CUDA install. Your CUDA install is somehow broken or incomplete. Since cuBLAS is required for dlib to use CUDA we won't use CUDA.
-- DID NOT FIND CUDA
-- Disabling CUDA support for dlib.  DLIB WILL NOT USE CUDA
-- C++11 activated.
Maybe need to purge previous CUDA first
https://gist.github.com/Mahedi-61/2a2f1579d4271717d421065168ce6a73#file-cuda_installation_on_ubuntu_18-04-L60
Step 23: tensorrt jackpot
https://developer.download.nvidia.cn/compute/machine-learning/repos/ubuntu1604/x86_64/



Step 24: None of the packages are installed… so not removed
Package 'nvidia-driver-390' is not installed, so not removed
Package 'nvidia-driver-418' is not installed, so not removed
Package 'nvidia-driver-430' is not installed, so not removed
Package 'nvidia-driver-435' is not installed, so not removed
Step 25: RUN nvidia-smi just can’t be run within the Dockerfile
Step 26:
System reboot recommended here but docker seems to obviate the need in this scenario
https://stackoverflow.com/questions/39712359/how-do-you-install-something-that-needs-restart-in-a-dockerfile



Step 27: https://stackoverflow.com/questions/51282529/install-dlib-error-not-found-cublas-v2
Install cublas-dev (sudo apt install cuda-cublas-dev-9-0) solved it for me.
Update path for cuda or cublas or cudnn https://docs.nvidia.com/cuda/cuda-installation-guide-linux/index.html#post-installation-actions
Step 28: no use changing the path variable to include CUDA as well
RUN ls -l /usr/local/cuda/lib64/libcublas*
RUN ls -l /usr/local/cuda/bin/*
ENV PATH="/usr/local/cuda/bin:${PATH}"
ENV LD_LIBRARY_PATH="/usr/local/cuda/lib64:${LD_LIBRARY_PATH}"
Step 29: cuda-drivers apt-get is responsible for nvidia version mismatch nvml and not even helping with dlib getting to use CUDA


Step 30: might have to drop tf:gpu-2.4.0 which keeps installing cuda-drivers-470 for apt-get cuda-11-0 and can’t find cuda-11-0-167 even in online cuda download site
Step 31: mention these deprecation warnings in presentation
/usr/local/cuda/include/cuda_runtime.h:1188:55: warning: ‘cudaError_t cudaBindTextureToArray(const texture<T, dim, readMode>&, cudaArray_const_t, const cudaChannelFormatDesc&) [with T = unsigned char; int dim = 2; cudaTextureReadMode readMode = (cudaTextureReadMode)1; cudaError_t = cudaError; cudaArray_const_t = const cudaArray*]’ is deprecated [-Wdeprecated-declarations]
   return err == cudaSuccess ? cudaBindTextureToArray(tex, array, desc) : err;
                                 ~~~~~~~~~~~~~~~~~~~~~~^~~~~~~~~~~~~~~~~~
/usr/local/cuda/include/cuda_runtime.h:1141:53: note: declared here
 static __CUDA_DEPRECATED __inline__ __host__ cudaError_t cudaBindTextureToArray(
                                                     ^~~~~~~~~~~~~~~~~~~~~~
What settled it in the end for me to go for tf:gpu-2.4.0 instead of tf:gpu-2.2.0 was there were more cuda deprecation warnings especially colmap ones in the end and el capitan my server wouldn’t support more than cuda 11.0 
Only three drawbacks are no nvidia-smi nvml version mismatch and CUDA warnings of:
W: Target Packages (Packages) is configured multiple times in /etc/apt/sources.list:50 and /etc/apt/sources.list.d/cuda.list:1
And dlib and colmap’s
-- Automatic GPU detection failed. Building for common architectures.
-- Autodetected CUDA architecture(s): 3.5;5.0;5.2;6.0;6.1;7.0;7.5;8.0;8.0+PTX
And the best thing we can do about it is to delete the duplicate lines in these file or safely ignore the warnings
A manual install of CUDA 11.0 using https://www.tensorflow.org/install/gpu would automatically install 11.4 as the instructions for deb(network) are identical to https://developer.nvidia.com/cuda-11.0-download-archive?target_os=Linux&target_arch=x86_64&target_distro=Ubuntu&target_version=1804&target_type=debnetwork 
Final successful log file for mpi:mark22 tf:gpu-2.4.0
mpi-mark22-build-log-202108171041.txt
  

Step 32: For those who really want to know why the version mismatch happened and how to prevent it from happening again. This is because of the versions of nvidia-* are different in these locations:
dpkg -l | grep nvidia (look at nvidia-utils-xxx package version), and
cat /proc/driver/nvidia/version (look at the version of Kernel Module, 460.56 - for example)
The restart should work, but you may want to forbid the automatic update of this package by modifying /etc/apt/sources.list.d/ files OR (I just found an easier way to hold the package) by executing this command apt-mark hold nvidia-utils-version_number.
Cheers.
P/S: Some contents were inspired by this (the original instruction was in Chinese, so i referenced the translated version instead)
https://stackoverflow.com/questions/43022843/nvidia-nvml-driver-library-version-mismatch
https://github.com/NVIDIA/nvidia-docker/wiki/Frequently-Asked-Questions#how-do-i-install-the-nvidia-driver



Step 33: COLMAP finally
git clone https://github.com/au001/view-synthesis.git
Cloning into 'view-synthesis'...
Username for 'https://github.com': au001
Password for 'https://au001@github.com': 
remote: Support for password authentication was removed on August 13, 2021. Please use a personal access token instead.
remote: Please see https://github.blog/2020-12-15-token-authentication-requirements-for-git-operations/ for more information.
https://docs.github.com/en/github/authenticating-to-github/keeping-your-account-and-data-secure/creating-a-personal-access-token
python3 ../view-synthesis/colmap/make_all_sparse_reconstructions.py train/
python3 make_all_sparse_reconstructions.py ../../../RealEstate10K/train2/


python3 ../view-synthesis/colmap/make_all_sparse_reconstructions.py train/
you can try doing ctl+z
and then find its process id using pid
and then do kill <pid>
and then fg
ctl-z does not kill it, just backgrounds i
sometimes you can also just hold down ctl-c for a long time?
git status
pull = fetch + merge.
You need to commit what you have done before merging.
So pull after commit.
check nvidia-smi every 0.5 seconds
watch -n0.5 nvidia-smi
exit watch by ctrl+c



Colmap is taking approx. 20 videos to process 1.5 hrs 
So a total of 67582 videos in train1 alone will take approx. 5068.5 hrs = 211 days
2 * 33719 videos in 2 * 105.5 days
4 * 16857 videos in 4 * 52.75 days
8 * 8428 videos in 8 * 26.375 days
16 * 4214 videos in 16 * 13 days
32 * 2107 videos in 32 * 6.5 days
Need to time commands https://linuxize.com/post/linux-time-command/
diff train-test-points/ train-test-points-multi-gpu/
Common subdirectories: train-test-points/points and train-test-points-multi-gpu/points
Common subdirectories: train-test-points/reco and train-test-points-multi-gpu/reco
Absolutely no difference between CPU and GPU COLMAP except for speed


COLMAP without GPU with time cmd for 3 videos
real    20m16.392s
user    105m32.380s
sys     13m58.531s
  COLMAP with multi-GPU with time cmd for 3 videos
real    10m16.757s
user    44m50.834s
sys     6m43.126s
COLMAP without GPU with time cmd for 3 videos
real    20m6.740s
user    104m20.324s
sys     13m39.825s


 Split the dataset by moving x number of files from the tail of the dataset
https://www.unix.com/shell-programming-and-scripting/142539-move-first-1000-files.html
ls | head -1000 | xargs -I{} mv {} ../temp


COLMAP is only using all the GPUs at once and not able to recognize intended GPUs
make_sparse_reconstruction.sh: 11: make_sparse_reconstruction.sh: --SiftExtraction.gpu_index=0,1: not found
make_sparse_reconstruction.sh: 21: make_sparse_reconstruction.sh: --SiftMatching.gpu_index=0,1: not found
Solution is to use --gpus flag to specify particular gpus to use
COLMAP errors
FilterH:        an illegal memory access was encountered
FilterV:        an illegal memory access was encountered
FilterH:        an illegal memory access was encountered
FilterV:        an illegal memory access was encountered
FilterH:        an illegal memory access was encountered
FilterV:        an illegal memory access was encountered
FilterH:        an illegal memory access was encountered
FilterV:        an illegal memory access was encountered
PyramidCU::GenerateFeatureList: an illegal memory access was encountered
Processed file [189/189]
  Name:            000188.png
  ERROR: Failed to extract features.
Elapsed time: 0.195 [minutes]




COLMAP error 1
an illegal memory access was encountered
This probably caused by insufficient GPU
memory. Consider reducing the maximum number of features.
https://colmap.github.io/faq.html#feature-matching-fails-due-to-illegal-memory-access
COLMAP error 1 solution: ask prof Ventura which GPU to use
Now trying to use both El Capitan and Colab
nvidia-smi --query-gpu=gpu_name,gpu_bus_id,vbios_version --format=csv
https://nvidia.custhelp.com/app/answers/detail/a_id/3751/~/useful-nvidia-smi-queries 
El Capitan 
name, pci.bus_id, vbios_version
Tesla V100-PCIE-32GB, 00000000:21:00.0, 88.00.48.00.02
Colab Pro+
name, pci.bus_id, vbios_version
Tesla V100-SXM2-16GB, 00000000:00:04.0, 88.00.9F.00.01
nvidia-smi --query-gpu=timestamp,name,pci.bus_id,driver_version,pstate,pcie.link.gen.max,pcie.link.gen.current,temperature.gpu,utilization.gpu,utilization.memory,memory.total,memory.free,memory.used --format=csv -l 5





All tf:gpu images have libnvinfer including 2.2.0-2.6.0
Check libnvinfer and libcudnn version on the image layers page 
https://hub.docker.com/layers/tensorflow/tensorflow/2.2.0-gpu/images/sha256-3f8f06cdfbc09c54568f191bbc54419b348ecc08dc5e031a53c22c6bba0a252e?context=explore
Latest tags for docker hub images and docker local images are being deprecated so I can take it easy with them 
docker pull tensorflow/tensorflow:devel-gpu
Was pushed literally 5 hours ago now 12:41 PM 8/19/2021
And docker pull tensorflow/tensorflow:2.6.0-gpu literally 8 days ago
Compatibility with newer cuda driver in image is possible with older kernel drive but it is a bit involved process so for future work
https://docs.nvidia.com/deploy/cuda-compatibility/index.html
Proving to be difficult to find the x in 11.0.x on El Capitan installed cuda 
Maybe we can just use the 2.6.0 with CUDA 11.2 right now because of the above link that states
CUDA 11.0 was released with an earlier driver version, but by upgrading to 450.80.02 driver as indicated, minor version compatibility is possible across the CUDA 11.x family of toolkits.
But got the same Dlib error that 
-- Found CUDA: /usr/local/cuda (found suitable version "11.2", minimum required is "7.5") 
-- Found CUDA, but CMake was unable to find the cuBLAS libraries that should be part of every basic CUDA install. Your CUDA install is s
omehow broken or incomplete. Since cuBLAS is required for dlib to use CUDA we won't use CUDA.
-- DID NOT FIND CUDA
-- Disabling CUDA support for dlib.  DLIB WILL NOT USE CUDA
And docker build fails with the usual COLMAP error
/colmap/src/mvs/gpu_mat.h:41:10: fatal error: curand_kernel.h: No such file or directory
 #include <curand_kernel.h>
As seen in /data/auppulur/mpi/mpi-mark25-build-log-202108191309.txt



Strangely in the tf:gpu-2.6.0 docker image I get a lesser nvcc version
Step 13/14 : RUN /usr/local/cuda/bin/nvcc --version
 ---> Running in e253c19591af
nvcc: NVIDIA (R) Cuda compiler driver
Copyright (c) 2005-2021 NVIDIA Corporation
Built on Sun_Feb_14_21:12:58_PST_2021
Cuda compilation tools, release 11.2, V11.2.152
Build cuda_11.2.r11.2/compiler.29618528_0
Than in El Capitan
auppulur@elcapitan:/data/auppulur/mpi $ /usr/local/cuda/bin/nvcc --version
nvcc: NVIDIA (R) Cuda compiler driver
Copyright (c) 2005-2021 NVIDIA Corporation
Built on Wed_Jul_14_19:41:19_PDT_2021
Cuda compilation tools, release 11.4, V11.4.100
Build cuda_11.4.r11.4/compiler.30188945_0



The main reason I think Dlib or COLMAP can’t find cuda and enable GPUs is 
!cat /usr/local/cuda/version.txt && !cat /usr/include/cudnn.h | grep CUDNN_MAJOR -A 2 
Are available in Colab but not in the docker containers of tg:gpu-2.2.0-2.6.0
And turns out CUDA version in Colab is actually 11.0.228 whereas I would have expected it to be some
So perhaps I need to manually install CUDA from https://developer.nvidia.com/cuda-11.0-download-archive?target_os=Linux&target_arch=x86_64&target_distro=Ubuntu&target_version=1804&target_type=deblocal
So if I just do an apt-get install cuda without manually also installing CuDNN and tensorRT then Dlib can’t enable CUDA due to missing CuDNN 
More importantly I’m not getting an NVML version mismatch when I don’t install CUDA manually
Maybe I gotto install the 450 El Capitan kernel driver as well and it depends on 460 that will automatically be installed too — but actually tf:gpu-2.2 deb (local) / runfile (local) specifically installs 460 so there is no need to download any driver separately but just do the deb (local) / runfile (local) installation rather than the deb(network) installation — maybe it's time to try runfile (local) rather than deb (local) or not
Deb (network) installation from 
https://developer.nvidia.com/cuda-11.2.1-download-archive?target_os=Linux&target_arch=x86_64&target_distro=Ubuntu&target_version=1804&target_type=debnetwork# https://www.tensorflow.org/install/source
Results in multiple versions of CUDA 11.2 and 11.4 and nvcc 11.4 and 450 and 470 drivers but only version of CuDNN 8.1.0.77-1+cuda11.2 and one version of libnvinfer 7.2.2-1+cuda11.1 being present as shown in /data/auppulur/mpi/deb-network-install-cuda-11-2-1.log  
  

describe trainig and inference pipeline to brng in openface
  
explain OpenFace in implementation

highlevel undersatnding of openface 

get head pose
