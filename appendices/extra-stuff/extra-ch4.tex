Results section:
Here are some evaluations that have been brought up with comparisons with baseline people depth estimation algorithms like MiDAS:

PSNR and SSIM metrics are computed on all pixels during evaluation.

At test time, 2020 MPI uses the point set to compute the scale factor sigma in the same way as they do
during training.

Steps to generate SSMI and PSNR
\cite{wang_ssmi_psnr}
\url{https://www.tensorflow.org/api_docs/python/tf/image/psnr}
\url{https://www.tensorflow.org/api_docs/python/tf/image/ssim}

% \newcolumntype{L}{>{\raggedleft\arraybackslash}m{3.5cm}}
% \newcolumntype{M}{>{\raggedright\arraybackslash}m{2cm}}
% \newcolumntype{N}{>{\centering\arraybackslash}m{1.5cm}}
% \newcolumntype{O}{>{\centering\arraybackslash}m{3cm}}

% \begin{table*}[t]
% \begin{sidewaystable*}[t]
%     \centering
    % \begin{tabular}{ML|NN|NN}
    % \begin{tabular}{M|M|M|M|M|M|M|M|M}
    % \toprule
    
    % & & \multicolumn{2}{O}{\textbf{LPIPS $\downarrow$} target\_image vs rendered\_image} & \multicolumn{2}{O}{\textbf{LPIPS $\downarrow$} reference\_image vs target\_image} \\
  
    % \textbf{Model Variant} & \textbf{Depth Loss Weight} & \textbf{Number of Disparity Map Channels Specified} & \textbf{Minimum Number of Visible Points per Frame} & \textbf{Steps Trained for} & \textbf{PSNR $\uparrow$} target vs rendered & \textbf{SSIM $\uparrow$} target vs rendered & \textbf{LPIPS $\downarrow$} target vs rendered \\
    
    % \cmidrule(lr){3-4} \cmidrule(lr){5-6}
    
    % \textbf{Model Variant} & \textbf{Dataset(s) (re)trained on / No. of Videos} & n = 5 & n = 10 & n = 5 & n = 10 \\
    % \midrule
    
    % Pretrained & RealEstate10K / $\sim$70k & 0.418 & 0.525 & 0.446 & 0.555 \\

    % \cmidrule(lr){1-2} \cmidrule(lr){3-4} \cmidrule(lr){5-6}
    
    % Recreated & Mannequin Challenge / 1841 & 0.319 & 0.433 & 0.446 & 0.555 \\
    
    % \cmidrule(lr){1-2} \cmidrule(lr){3-4} \cmidrule(lr){5-6}
    
    % Recreated  & Mannequin Challenge + RealEstate10K & 0.308 & 0.412 & 0.466 & 0.555 \\
    
    % \cmidrule(lr){1-2} \cmidrule(lr){3-4} \cmidrule(lr){5-6}
    
    % Recreated multi-GPU & Mannequin Challenge & 0.418 & 0.525 & 0.446 & 0.555 \\
    
%     \bottomrule
%     \end{tabular}
%     \caption{LPIPS Mean Values}
%     \label{tab:lpips}
%     {\small n refers to the distance between the reference and target frames picked by the generator. Retraining promises marked improvement over original pretrained model.}
% % \end{table*}
% \end{sidewaystable*}

A further testimony to this improvement can be obtained by inspecting the performance of even the prematurely halted multi-GPU variant. It performs at par with the original pretrained model which indicates that the pretrained model has begun to continue where it left off and specialize in processing video-chat-like frames. It would have run properly if not for the resource errors mentioned in the earlier that could point to underlying issues like possible unchecked growth of TensorFlow graphs per pipeline replica or such. This seems to be the case even though the replicas seem to be getting properly allocated inputs and their respective outputs also seem to be getting well gelled together in the end.