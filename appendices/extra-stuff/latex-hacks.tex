An alternative to this approach, which facilitates the production of nested lists is the outlines package. To produce a bulleted list with three levels it is as simple as
% \documentclass{article}
% \usepackage{outlines}
% \begin{document}
% \begin{outline}
%  \1 Top level item
%   \2 Sub item
%      \3 sub sub item
% \end{outline}
% \end{document}

To make a numbered list (as opposed to a bulleted list) one can simply pass the enumerate option to this package
% \documentclass{article}
% \usepackage{outlines}
% \begin{document}
% \begin{outline}[enumerate]
%  \1 Top level item
%   \2 Sub item
%      \3 sub sub item
% \end{outline}
% \end{document}

% Notice that to insert the parentheses or brackets, the \left and \right commands are used. Even if you are using only one bracket, both commands are mandatory. \left and \right can dynamically adjust the size, as shown by the next example:

\[ 
 \left[  \frac{ N } { \left( \frac{L}{p} \right)  - (m+n) }  \right]
\]

% ctrl+i for automatic \textit insertion
% ctrl+b for automatic \textbf insertion