

https://mashable.com/video/google-project-starline-3d-video-calls


Project Starline is the name of the new prototype machine for face-to-face meetings. The term "video booth" truly captures the essence of Starline in its current incarnation: It's a huge booth similar to those found in diners, but much more technologically sophisticated. The Project Starline booths feature a variety of depth sensors and cameras. These sensors collect photorealistic, three-dimensional imagery; the system compresses and transfers the data with apparent low latency to each light field display on both ends of the video discussion. All data is delivered using WebRTC, the same open-source technology that powers Google Meet, the company's primary video conferencing application. Google will aim to scale back Project Starline as it perfects the technology.



https://ieeexplore.ieee.org/document/9105988
This paper has a similar theme as the current thesis but differs mainly, in that, it has it renders 3D video using an intermediate light field representation as opposed to the current Multiplane Representation

https://www.sciencedirect.com/science/article/pii/S2090447919301297



Google has spent the last several years developing software experiences that make you feel as though you're in the same room as another human person, even if they're many time zones away. On the one hand, there's dull Google Meet, the company's Zoom competitor. He desired photo-realistic, volumetric video meetings that looked, sounded, and felt just like the other person was sitting across the table from you—without the use of a headset.

There was the Starline booth, which was partially wood-paneled and partly covered in gray cloth, and had an integrated bench on one side and a 65-inch display on the other.
You could be forgiven for believing that Starline was created during the pandemic, while desk employees were umm-ing, muting, and unmuting their way through an endless stream of Meets and Zooms.
Bavor asserts that there was no "aha" moment that precipitated the creation of Project Starline.
“What has always fascinated me about virtual and augmented reality is the idea that these technologies can transport you to other locations and make you feel as if you are physically present in another location,” Bavor explains.
“However, it did not appear as though there was a method to bring the most important things in the world to you, specifically the people you care about.”
For several years, Starline will remain a notion, unlikely to enliven your Google Meet meetings.
Andrew Nartker, Project Starline's lead product manager, has been bringing an apple to meetings.
It's a method of demonstrating how objects in Project Starline interact, and, perhaps more disturbingly, a method of tracking eyeballs.
“I can show you this Whole Foods apple and see precisely what you're looking at,” Nartker adds as I take a seat in the booth.
Nartker is conferencing in from a second identical Project Starline booth to the one being used by the author.
The author is currently seeing a 65-inch light field display.
(When the author requests specifics about the equipment, Google is evasive.)
Google has not disclosed the cost of constructing a Project Starline booth; my best guess is "quite a bit." Project Starline, on the other hand, is a condensed version of much bigger volumetric capture studios.

Project Starline is unlikely to find a home in your impromptu home office anytime soon.

While it's difficult to imagine this type of technology operating seamlessly over a shoddy home Wi-Fi connection—Google did confirm that the author's booth was hardwired to the building's network—one of the Project Starline engineers insisted that the technology would work using Google's standard-speed office network, without the need for fiber.
In Project Starline, the author encountered three distinct Googlers, and part of the surreality dissipated as the author shifted in his or her seat.
When Nartker began projecting a web page onto the light field display as a demonstration of how two individuals may communicate in Starline, the authors simply peered over each other's right shoulders at a non-interactive page.
According to Google, perhaps a hundred employees have utilized Starline, which has been hidden in secret offices in Mountain View, Seattle, and New York.
Bavor has been utilizing it for the majority of his recent meetings in Seattle and New York, spending approximately 50 hours in the booth.
He says that his encounters in Starline have left his brain with a stronger brushstroke, that he has better recollection of details, and that he leaves meetings with the impression of having met the individual.
“I'm aware that the individual across from me is not checking his phone during the discussion, which is nice,” Bavor says.
In this context, Google's Project Starline appears particularly overengineered, a synthesis of accessible technology (Google Meet), geek technology, and a meticulously constructed, immobile small studio, all for the sake of... more video meetings.
If you create an article implying that Google Glass is no longer available, the firm's public relations team will swiftly inform you that the company continues to sell a product called Glass Enterprise Edition 2.
One has to wonder if Bavor is less interested in head-up displays these days, given that it appears as though every other consumer technology company is developing face computers.
Bavor maintains that VR is “extremely powerful in its potential to transport you to another place,” and that there is a connection between AR and VR and Project Starline.

Bavor said Google will conduct trials of the technology later this year with a select group of early adopters, including enterprise cloud organizations, telemedicine apps, and media companies, though he declined to name them.
The Project Starline booths will be largely utilized by Googlers entering workplaces, who will marvel at the realism, hold up their apples, and briefly overlook the gap between realism and reality.
Lauren Goode is a senior writer at WIRED, where she covers goods, applications, services, as well as consumer technology news and trends.
