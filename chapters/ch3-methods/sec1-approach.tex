\chapter{Methods}\label{ch3:methods}

The objective of this work has been to freely rerender concurrent one-on-one video chat frames from the points of view of both participants bidirectionally and in real-time. This would help simulate the experience of conversing face-to-face with a person in the real world. We adopted Tucker and Snavely's~\cite{single_view_mpi} single-view MPI network, for it is the first state-of-the-art open-source single-view view synthesis network, and it has been quite popular among various enterprises and organizations since its release in 2020. When we initially ran the publicly available inference part of the network on a video chat frame, we found that the generated disparity map (Equation~\ref{eq:disparity-map}) was visually inaccurate. Comparatively (Figure~\ref{fig:great-off-kilter-disparity}), the inferred disparity map would be much more visually accurate whenever a real estate video frame would be processed. The latter outcome is to be expected because Tucker and Snavely's model was originally trained on RealEstate10K~\cite{zhou2018stereo} video dataset. Specifically, certain aspects of the synthesized views, such as image sharpness, would be pretty compelling for the real estate category of video frames by virtue of the model having been efficiently tweaked and extensively trained by the authors (given contemporary hardware limitations). Yet, synthesized video-chat-related frames alone would seem unnaturally concave/convex at arbitrary positions within each rerendered frame, not to mention the loss of perspectivity and the induction of random distortions occurring within the frame as well.

\begin{figure}[!h]
    \includegraphics[width=1\columnwidth]{figures/great-off-kilter-disparity.png}
    \caption{Disparity Heat Maps Synthesized by Tucker and Snavely's Model~\cite{single_view_mpi} for Real Estate and Video Chat Frames}
    \label{fig:great-off-kilter-disparity}
    {\small The disparity map on the left encodes a real estate scene, and the one on the right encodes a video chat scene. The real estate map successfully shows appropriate heat/depth gradations from the hottest/closest armrest region on the bottom right to the coldest window regions toward the back of the scene. The video chat map, on the other hand, counterintuitively shows that the face of the girl in the scene is situated behind the body, and the couch in it is somehow disjointed.}  
\end{figure}

\section{Approach}\label{sec:approach} 

As a primary step (Figure~\ref{fig:mpi-training-pipeline}), we attempted to increase Tucker and Snavely's depth prediction accuracy for video-chat-relevant frames containing close-up shots of people so that we may see a drastic reduction in the number of artifacts induced in synthesized frames. This involved curating and utilizing both RealEstate10K and MannequinChallenge~\cite{li2019learning} datasets. The latter contains video frames that resemble video chat scenes: it is composed solely of scenes of people pretending to be mannequins while a camera moves around them, flowing seamlessly from scene to scene. Essentially, we performed transfer learning~\cite{radhakrishnan_what_2019} with the pretrained weights of Tucker and Snavely's model, by \textit{fine-tuning}/\textit{refitting} them to a dataset other than the one they were originally trained on. Secondly (Figure~\ref{fig:3d-video-chat-rendering-pipeline}), we introduced the head pose detection submodule of OpenFace 2.2~\cite{baltrusaitis_openface_2018} into the inference pipeline of Tucker and Snavely, so that ``\textit{viewee}" video frames may be rerendered at the head pose obtained from ``\textit{viewer}" frames. We considered a few state-of-the-art open-source head pose estimation models, including WHENet~\cite{zhou_whenet_2020} --- for its speed and consistency. We ultimately chose OpenFace 2.2 because it works well with the Deep Learning (DL) framework used by Tucker and Snavely (TensorFlow 2.2) and can be installed in the same dockerized environment as COLMAP~\cite{schoenberger2016sfm,schoenberger2016mvs} and the rest of the dependencies needed by our comprehensive pipeline. 

\begin{figure}[!h]
    \includegraphics[width=1\columnwidth]{figures/mpi-training-pipeline.png}
    \caption{MPI Training Pipeline}
    \label{fig:mpi-training-pipeline}
\end{figure}

Out of the non-exhaustive set of network components made publicly available by Tucker and Snavely~\cite{single_view_mpi}, a comprehensive inference pipeline on Google Colaboratory (Section~\ref{sec:code-sources}) was one. It immensely helped us with our OpenFace integration and gave us the ability to visualize and present our results and demos in chapter~\ref{ch4:experiments-results} and everywhere else. They couldn't reveal certain other aspects of their codebase due to their proprietary natures. This prompted us to go about recreating Tucker and Snavely's DispNet-like model~\cite{mayer_large_2016} first before retraining it on requisite datasets and repurposing it for video chat view synthesis. We recreated parts of the model from the code released (Section~\ref{sec:code-sources}) by the authors involving their network definition (convolutional layers, kernel sizes, etc.) and the code used by them for rendering views from new camera positions with homographies and related operations (Equation~\ref{eq:rerendered-target}). We then put together other aspects of the network that called for a more involved recreation process, like the data loader part and the loss functions (Equation~\ref{eq:aggregate-loss}). Requisite components of input data, including point clouds, had to be extracted and loaded in. One of the key features of Tucker and Snavely is to use sparse point cloud data to make the view synthesis loss scale-invariant (Subsection~\ref{subsec:base-papers}). To obtain such inputs, we processed both datasets with COLMAP and wrote a custom data loader. We took inspiration from Zhou et al.'s~\cite{zhou2018stereo} stereo MPI paper for building the data loader, for the code they tailored to load in data (Section~\ref{sec:code-sources}) was refactored and reused by Tucker and Snavely as well. Their implementations of subsequence selection and random cropping proved pretty useful.

\begin{figure}[!h]
    \includegraphics[width=0.60\columnwidth]{figures/3d-video-chat-rendering-pipeline.png}
    \caption{3D Video Chat Rendering Pipeline}
    \label{fig:3d-video-chat-rendering-pipeline}
\end{figure}

We retrained the recreated network in two different ways. One group of model variants was fine-tuned exclusively on the video-chat-relevant MannequinChallenge video dataset~\cite{li2019learning}, which is $\sim$96\% smaller than RealEstate10K in training data as of this writing. The other set of variants was retrained on a combination of both datasets by having the model pick same-sized batches of training data (Subsection~\ref{subsec:base-papers}) randomly and alternatingly from both datasets. We considered addressing this inherent data imbalance problem by making the model pick an appropriate proportion of RealEstate10K frames for every MannequinChallenge frame randomly selected. However, we ultimately voted against it in favor of resolving more pressing issues such as the training errors mentioned in section~\ref{sec:implementation}. We are grateful to the authors of Tucker and Snavely for forewarning us that there is a risk of overfitting to the much smaller MannequinChallenge dataset, even though it was generally mentioned in both Zhou et al. and Tucker and Snavely that the stereo and single-view models were quite generalizable to domains besides real estate footage. Hence, we felt the need to deploy the second set of variants to help access this risk. We could also have taken another transfer learning route of freezing all but the last few layers of the model to possibly reduce overfitting, but we chose to unfreeze all layers in favor of making the variants wholly robust. The layers were thus free to learn and evolve based on the MannequinChallenge data they were newly exposed to. We stack these variants up against each other and also against the pretrained single-view model and compare their performances in chapter~\ref{ch4:experiments-results}. Finally, after introducing the head pose estimation API of OpenFace 2.2 into the inference pipeline of the variants, we converted estimated head orientations into a form amenable to rendering with MPIs. This involved manipulating yaw, pitch, and roll head angles, and the MPI helper functions provided by Tucker and Snavely went a long way in making this possible as well. We also visually verified for if the rerendered frames were getting seemingly aligned with the extracted head poses or not.

% \section{Data}\label{sec:data} 

The steps taken to get both Mannequin Challenge~\cite{li2019learning} and RealEstate10K~\cite{zhou2018stereo} datasets ready for training are as follows:

\begin{itemize}

    \item Both these datasets consist of text files pertaining to each video. Each text file begins with the downloadable video’s YouTube link on the first line. And, from the second line onward, it continues with listing the details of ORB-SLAM2 and COLMAP processed frames from the video \footnote{with one line for each frame} --- such as the timestamp (in microseconds), camera intrinsics, and camera extrinsics.
    
    \item Each dataset from the URLs listed in \cite{zhou2018stereo} and \cite{li2019learning} was downloaded. 
    
    \item The GitHub repository associated with this thesis was downloaded from \url{https://github.com/au001/view-synthesis.git} into the working directory.

    \item The project's comprehensive Dockerfile was built from within the cloned repository by running \textit{build-run-docker/build-docker.sh}. Considerable time was spent to resolve dependency-version compatibilities by consulting the build log file whenever Dockerfile failed to build. After a successful build, the final docker container was started with \textit{build-run-docker/build-docker.sh}.
    
    \item {\sloppy Back inside either downloaded dataset directory, we ran the script \textit{../view-synthesis/scripts/get\_videos.py} on \textit{train/} folder to download all videos with youtube-dl at 720p resolution. Alternatively run \textit{../view-synthesis/scripts/get\_videos\_aria2c.py} to bolster youtube-dl’s download speed with Aria2 download manager. Standard output was saved to a log file to address possible download errors.}  
    
    \item Inevitably youtube-dl would throw download errors on the first run as there would be some partial and/or skipped downloads for various reasons ranging from the videos being taken down from YouTube over time to fixable errors intrinsic to youtube-dl. Moreover, some videos were unavailable in their 720p versions and were with the aim of maintaining consistency. Although scaled videos should not pose any problem to the training or the 3D point generation with COLMAP, we proceeded to attempt something different from the 2020 MPI authors, assuming the permitting scaled videos in their pipeline.
    
    \item As of this writing 2663 of $\sim$3000 Mannequin Challenge videos and 67582 of $\sim$70000 RealEstate10K videos were downloadable at 720p resolution and with no download errors on the first attempt. Hence, it became imperative to also save all the outputs generated by running the previous script to a log file and work upon it fix all download issues.
    of the available videos only so many were actually used colmap processed and used for training 
    The difference in numbers is attributed to availability of videos and COLMAP processing requirements not being met for some of them.

% Step 3: Extract to a new log file the lines from the previously saved log file that contain specific combinations of the strings “error:” and ": Downloading webpage" Run get_errors.py on get-vid-log-train-202106241546.txt
% python3 ../scripts/get_errors.py get-vid-log-train-202106241546.txt train-errors-202106281624.txt
% Step 3: Find the names of all the youtube online videos in <train-errors-202106281624.txt> that can be found in this specific pattern
% "youtube\] (.*?)\: Download"
% python3 ../scripts/get_name_within_pattern.py train-errors-202106281624.txt train-error-names-202106281635.txt
% Step 3 can only be executed if there are no .part mp4 downloads remaining in <train/> <test/> or <val/> folders
    
    
    
    % \item Similarities between MannequinChallenge and RealEstate10K datasets
    
    
\end{itemize}







% training we require triplets of images together with their relative
% camera poses and intrinsics.
% absolute camera poses are not required

% Visual SLAM and structure-from-motion have no way of determining absolute scale without external information:
% each of our training sequences is therefore equally valid
% if we scale the world (including the sparse point sets and
% the translation part of the camera poses) up or down by
% any constant factor. This is not an issue when dealing with
% multiple-image input since the relative pose between the
% inputs resolves the scale ambiguity, but it poses a challenge for learning any sort of 3D representation from a single input.

 
% /include{figure} training pipeline 
% draw.io


% inputs and outputs for all components 
% dataset text files dowload -----> youtubr downloader downloaded videos -------> COLMAP 3d points ------> 
%                                         |
%                                         |
%                                         |-------------------------------------------------------------->



Colmap is taking approx. 20 videos to process 1.5 hrs 
So a total of 67582 videos in train1 alone will take approx. 5068.5 hrs = 211 days.

% Descriptions of Mannequin Dataset, Real Estate dataset, COLMAP processing pipeline, data loader ---

% COLMAP
There is a correspondence between the camera location C in world coordinates the distance lambda from C to the observed 3D world point in the scene along the viewing ray of a projected pixel x on the imaging. 

COLMAP is a 3D scene reconstruction pipeline. It attempts to recover the 3D scene structure from unstructured 2D images of the scene that come with no prior knowledge of the camera intrinsics, extrinsics, and nature of objects captured in the image. The extracted scene structure is either in the form of sparse 3D points along with the camera parameters for each input 2D image or in the form of dense 3D points with associated color information.

feature detection --- pairwise feature matching --- correspondense estimation --- incremental structure from motion

We had to make sure that we subject videos to COLMAP processing only after ensuring that their 720p version of them were downloaded and for videos that were missed during handling of these large datasets we had to make sure that after properly redownloaded the video again we COLMAP processed it again. 


% Expanding upon pipeline description ---

% video ------------> input 1 --------------> training
% |
% |
% |-----------------> COLMAP -----------> 3D points per video frame -------------> input 2 -----------> same training

% how does data loader work?

% Original authors choice with colmap 
% my reason for pursuing colmap
% what does colmap do> --------> bundle adjustment, triangulation, 

% COLMAP highlevel

% Triangulation
% Bundle Adjustment

% data section

% for the 2018 paper
% During training, the images and
% MPI have a spatial resolution of 1024 × 576, but the model can be
% applied to arbitrary resolution at test time in a fully-convolutional
% manner.
% but maintaining resolution for 2020 paper is not required 

% 4.3 Refining poses with bundle adjustment
% We next process each sequence at higher resolution, using a standard
% structure-from-motion pipeline to extract features from each
% frame, match these features across frames, and perform a global
% bundle adjustment using the Ceres non-linear least squares optimizer
% [Agarwal et al. 2016].

% errors in our retraining attempts despite the authors of the preexisting model indicating in their paper that the model could be trained on a hodgepohdge of multiple resolutions   















% \section{Implementation}\label{sec:implementation} 

We attempted to generate accurate MPI representations for close-up targets such as heads and upper bodies and improve the pixel accuracies of views synthesized from these MPIs. After putting together the data loader to feed the datasets and point clouds into the network, we recreated loss functions from the textual descriptions in the single-view MPI paper~\cite{single_view_mpi}. As mentioned in subsection~\ref{subsec:base-papers}, we likened our training process to Tucker and Snavely's~\cite{single_view_mpi}, with respect to various aspects such as using TensorFlow 2.2, ADAM solver, a pixel loss weight of 1, a smoothness loss weight of 0.5, etc. We experimented with choices of learning rate and depth loss weight but generally picked 0.00001 and 1, respectively, contrary to the 0.0001 and 0.1 used in Tucker and Snavely. We reduced the learning rate because we were fine-tuning the pretrained model rather than training from scratch. The requirement that we had to have view synthesis quality as supervision was fulfilled by taking frames one frame apart\footnote{in the video sequence} from each chosen training frame as target ground truth. We trained for a certain number of steps rather than for a certain number of epochs. This is because, generally, only smaller/faster-to-train datasets are used for training a model in epochs, whereas it is easier to train larger, indeterminate-in-size datasets in steps. Our data loader randomizes batch picking not only for testing but also for training. Moreover, we have not yet been able to go beyond the model experimentation stage. Exposing the model to a wide variety of frames is the way to go in this stage. For the model to be trained sequentially on all frames clip by clip, covering entire datasets multiple times in multiple epochs, it should be free of any errors that impede its progress toward convergence. We have not been able to bring our model up to that stage yet. 

We used wandb.ai~\cite{wandb} for experiment tracking. It proved to be a valuable tool for our entire process. It helped us spin out different model variants, chiefly characterized by their being trained either on MannequinChallenge alone or on a combination of both datasets. As with some notable attempts at model training in the community, we encountered Not a Number (NaN) gradient errors that took a good chunk of our resolution efforts in this work but ultimately could not be resolved. NaN losses signal that the issue of vanishing/exploding gradients may be present. In this work, NaN gradients could only be reduced in their frequency of occurrence from once in several hundred steps to once in several thousand steps. wandb.ai helped immensely in resuming not just the training runs themselves but also the activity of logging training metrics right from when the run broke off due to a NaN error. What also helped bring down the frequency of encountering NaNs, we believe, was the fact that we removed all those videos from the training/testing process that had at least one frame with a point cloud composed of less than two 3D points. Our Linux command to locate such point cloud \textit{.txt} files (Section~\ref{sec:code-snippets}) would take about 3 hours to sift through a set of 2500 point cloud directories with one \textit{.txt} file per video frame. Replacing \textit{cumprod} used in several places in the single-view MPI source code with \textit{safe\_cumprod}, as suggested to us by one of the authors of the single-view paper, also helped reduce the frequency of encountering NaNs. One of the issues we could completely resolve was the occasional throwing of \textit{ValueErrors} by our data loader. We also attempted to redress the rendered artifacts mentioned in section~\ref{sec:approach} and determine if real-time, high-quality view synthesis was indeed possible without game engines.

We used customized training loops with TensorFlow's \textit{tf.GradientTape} context~\cite{noauthor_custom_nodate}. Nevertheless, we found that the gradient calculation (Section~\ref{sec:code-snippets}) would take about a minute! We were using a batch size of 8 on an NVIDIA V100 GPU at the time. The authors of the single-view MPI paper, however, informed us that even on a single worker, their gradient calculation would take less than a second. They then astutely diagnosed our issue to be that we were doing everything in \textit{eager mode}, resulting in excessive overhead. They suggested that using Keras's \textit{model.fit}, using the old estimator system of TensorFlow, or just wrapping things in \textit{tf.function} should allow the critical parts to run in graph mode and be faster. They also suggested that things were probably too big to fit on our GPU. Also, the authors had used a batch size of 4. We ultimately adopted the use of \textit{tf.function} wrapper as well as a batch size of 4 and were able to complete implementing our training and testing pipelines.

\begin{figure}[!h]
    \includegraphics[width=0.75\columnwidth]{figures/openface-csv.png}
    \caption{A Snapshot of OpenFace 2.2~\cite{baltrusaitis_openface_2018} Outputs}
    \label{fig:openface-outputs}
\end{figure}

We then inserted OpenFace 2.2~\cite{baltrusaitis_openface_2018} into the inference pipeline of one of our better-performing model variants and attempted to emulate a video chat system, one half at a time. Using OpenFace 2.2, we extracted the head pose from each frame of a ``viewer" video sequence, as shown in figure~\ref{fig:3d-video-chat-rendering-pipeline}. We used one of the utility functions in the single-view MPI modules, \textit{geometry.pose\_from\_6dof}, to extract the yaw, pitch, and roll angles of the ``viewer" frames in a manner conducive to being accepted by the MPI inference. We then rendered the ``viewee" video sequence at the head pose of the ``viewer" frames with matching timestamps. Even though it looks like more precision could have been added by using not only head pose estimation but also gaze estimation with OpenFace, a compelling argument can be made to the contrary that when we look at people or at a scene, whatever we view does not seem to get ``rerendered" in our visual system based on our changing gaze. It seems to get ``rerendered" based (perhaps solely) on our changing head pose. A snapshot of OpenFace 2.2 outputs for multiple frames in a sequence is shown in figure~\ref{fig:openface-outputs}.

