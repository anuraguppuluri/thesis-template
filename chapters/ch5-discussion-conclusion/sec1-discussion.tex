\chapter{Discussion}\label{ch:discussion}

% from deep stereo paper the causes for overfitting

% Deep networks have enjoyed huge success in recent
% years, particularly for image understanding tasks [20, 29].
% Despite these successes, relatively little work exists on applying
% deep learning to computer graphics problems and especially
% to generating new views from real imagery. One
% possible reason is the perceived inability of deep networks
% to generate pixels directly, but recent work on denoising
% [35], super-resolution [6], and rendering [21] suggest
% that this is a misconception. Another common objection is
% that deep networks have a huge number of parameters and
% hence are prone to overfitting in the absence of enormous
% quantities of data, but recent work [29] has demonstrated
% state-of-the-art deep networks whose parameters number in
% the low millions, greatly reducing the potential for overfitting.

% from deep stereo on the success of neural nets

% In this work we present a new approach to new view synthesis
% that uses deep networks to regress directly to output
% pixel colors given the posed input images. Our system
% is able to interpolate between views separated by a
% wide baseline and exhibits resilience to traditional failure
% modes, including graceful degradation in the presence of
% scene motion and specularities. We posit this is due to the
% end-to-end nature of the training, and the ability of deep
% networks to learn extremely complex non-linear functions
% of their inputs [25].
% Additionally, although we focus on its application
% to new view problems here, we believe that the
% deep architecture presented can be readily applied to other
% stereo and graphics problems given suitable training data.
% Because
% of the variety of the scenes seen in training our system is robust
% and generalizes to indoor and outdoor imagery, as well
% as to image collections used in prior work.

Through this thesis, we have had the opportunity to simulate both halves of a 2-way pipeline that is able to render novel views from the perspectives of both participants in a video chat. Going by the synthesized monochromatic disparity maps and even by the SSIM values, we found that our model variants struggled to synthesize disparity well. Going by the LPIPS values, we found that they excelled at synthesizing the actual target view itself. This is indeed unexpected owing to the fact that only one of the given 32 MPI layers is able to essentially duplicate the reference image in its entirely.

A further testimony to this improvement can be obtained by inspecting the performance of even the prematurely halted multi-GPU variant. It performs at par with the original pretrained model which indicates that the pretrained model has begun to continue where it left off and specialize in processing video-chat-like frames. It would have run properly if not for the resource errors mentioned in the earlier sections (let's mention the section number here) that could point to underlying issues like possible unchecked growth of TensorFlow graphs per pipeline replica or such. This seems to be the case even though the replicas seem to be getting properly allocated inputs and their respective outputs also seem to be getting well gelled together in the end.

Although the sharpness of the rerendered images is almost twice as good with our chosen model variant as with the pretrained baseline, the predicted MPIs layers have all but collapsed to a single depth layer. This is also evident from the way the training would start to produce completely gray disparity maps from around step 14,000 onward, as noted in chapter~\ref{ch4:experiments-results}. We believe that the reason for this is more likely to be found in the weights we assigned to our various loss functions that aggregate into a mean loss. Ablation experiments involving taking out the pixel loss and/or bringing the smoothness loss way down would help to isolate the issue even more. Although we made our best efforts to reconstruct the loss functions and the rest of training setup as close to the textual descriptions in the paper as possible, it would definitely shine a lot more light on the root cause of the problem if we are able to access the training script of the authors --- something that they've had to keep from the public. Also, since we were also meticulous with our data curation, we don't believe it is likely that the input data has any part to play in the generation of NaN loss errors.

One of the obvious next steps would be to perform \textit{hyperparameter sweeps} with wandb.ai to find optimal hyperparameters, including the weights of the loss functions, and potentially solve the vanishing/exploding gradients problem which could very well be related to the issue of the swiftly saturating disparity maps. If we are actually able to get the plan to work, it would reveal why the pretrained model found it hard to synthesize disparity for video chat frames in the first place: it was not exactly as generalizable as the authors hinted it might be. But, if after running all possible hyperparameter sweeps with something like wandb.ai, we still find that the model performs poorly, then the obvious next thing to look at would be the actual training scripts used by the authors to discover how way off the mark we could have been in replicating their network.

% \chapter{Conclusion}\label{ch:conclusion}
\section{Conclusion}\label{sec:conclusion}


In this thesis, we have not created novel models or datasets but have rather curated preexisting datasets and retrained a state-of-the-art CNN. Data curation has been an essential part of our work as the datasets' YouTube videos are subject to modifications over time. These modifications are in terms of the videos being taken down from YouTube or the required 1280$\times$720 pixel (720p) resolution versions of them becoming unavailable, etc. The curation process included action items like downloading and training only on 720p versions of the datasets' videos so as to minimize the chances of running into training errors, etc., as explained in section~\ref{sec:data}. As for simulating the 3D video chat experience itself, we linked-up the API of OpenFace 2.2~\cite{baltrusaitis_openface_2018} --- a preexisting head pose estimation model --- to the MPI inference procedure so the MPI inference may generate novel views rendered in the head pose evaluated by OpenFace 2.2, as explained in section~\ref{sec:implementation}.

We used MPI to solve 3D video chat problem because of it real-time view and other view synthesis properties mentioned in the base paper section above. High quality, spatially-consistent, and high-resolution synthesis by rendering with MPIs which are essentially mini-local-light-field representations have been accomplished.
% in defence mention that MPI essentially constructs a mini light field.
\section{Future Work}\label{sec1:future-work}
Maybe we could implement taking the average of the head poses of multipe people in the video frames of video conferences instead of just video chat and make their average head pospe change the scene rendering viepoint of teh secne to be rerenered.

Through this thesis, I had the opportunity to form a 2-way pipeline that is able to render new views from the perspectives of both the participants in Video Chat conversation.

Increase the training speed of the MPI model by making it a multi-GPU model with the constantly-evolving, cutting-edge tf.distribute.Strategy API for distributed training with TensorFlow/Keras.
This would allow for feeding a lot more images/video-frames to the model, which would further reduce the accuracy of the model.

Using Grad-CAM to locate the bottlenecks in the recreated MPI neural net to optimize hyperparameter tuning for producing more accurate results, esp. predicted depths / disparity.
https://www.pyimagesearch.com/2020/03/09/grad-cam-visualize-class-activation-maps-with-keras-tensorflow-and-deep-learning/


Render in both directions, making the pipeline two-way and then proceed to make it realtime by involving a game engine or any other framework capable of realtime rendering. 

Try training on variable resolution video frames and not all just 1280x720 ones

Ideal for a headless server?

make use of docker multistage builds to have all components in a single dockerfile
with multiple froms like tf/tf-gpu-2.2 as well as nvidia-cuda10.2-devel-ubuntu18.04

possible hypothesis: did cuda support improve disparity map

possible hypothesis: cuda gpu support is possibly not required for OpenFace inference

features of OPenFace like head pose estimation may still work without CUDA recognition by the server either COlab or El Capitan 

need gpu support for maybe mpi training alone and not any other components of the pipeline as the rest of the pipeline is just inference

COLMAP will be quicker with GPU
https://colmap.github.io/faq.html#available-functionality-without-gpu-cuda

major hypothesis: one major reason with disparity map to be less because the batch size was only 4 frames at once   
if multi-gpu access were available then disparity map would have been better 

Mainly mpi and maybe even colmap (for inference speed) seem to require GPU/CUDA support. I've been trying to get GPUs to be used by all my packages on Docker like MPI, COLMAP and their dependencies OpenCV, Dlib etc.
I doesn't seem to work yet. So I'll resort to using these packages on Docker without GPU support for now. 

in videos we have recording of my insights today - 8/15/21
been falling behind and didn't report until resukts 

and update to flagship versions

the main thing Dr ventura is that the CUDA install was broken and I needed it for multiple programs like dlib, openface, notwithstanding colmap 

ask prof ventura to update the nvidia drivers 

why did i go off on a tangent?
https://stackoverflow.com/questions/43022843/nvidia-nvml-driver-library-version-mismatch
Available functionality without GPU/CUDA

https://colmap.github.io/faq.html#available-functionality-without-gpu-cuda
If you do not have a CUDA-enabled GPU but some other GPU, you can use all COLMAP functionality except the dense reconstruction part. However, you can use external dense reconstruction software as an alternative, as described in the Tutorial. If you have a GPU with low compute power or you want to execute COLMAP on a machine without an attached display and without CUDA support, you can run all steps on the CPU by specifying the appropriate options (e.g., --SiftExtraction.use_gpu=false for the feature extraction step). But note that this might result in a significant slow-down of the reconstruction pipeline. Please, also note that feature extraction on the CPU can consume excessive RAM for large images in the default settings, which might require manually reducing the maximum image size using --SiftExtraction.max_image_size and/or setting --SiftExtraction.first_octave 0 or by manually limiting the number of threads using --SiftExtraction.num_threads.

nvidia-smi
Failed to initialize NVML: Driver/library version mismatch


what is cuda and nvcc all about?
https://varhowto.com/check-cuda-version/

dockerfile colmap run needs to be explained with video clip in MAnnequinChallenge

proof that gpu is being used by colmap in images in MannequinChallenge


https://linuxize.com/post/linux-time-command/
time all functions

make sure I'm able to restart the model at any poit and continue traning whre I left off and add datasets 

I need info on inference code 

Ask ventura why did yoyu say the disparity was bad

Hopefully successfully able to use the latest versions of all components of the mpi pipeline for both training and inference  

Another application would be if we have a VR headset with a camera on it we can track the rotation of the camera and by doing that you're tracking the rotation of the person's head so that you can render the VR content at the right angle

Ask prof ventura is colmap autmatically redoes all error videos 
Ask pro ventura about copyright for his own epipolar geometry lectures

stereo = 2 images pretty close to each other paired in a special way so that you can get really dense estimations of the depth so basically for every pixel you could get a depth estimate rather than some sparse sampling of keypoints
canonical stereo case only have pure horizontal translation and no rotation and no other translations in Y or Z  

blueer values are closer and redder values are farther away in disparity maps

check 7000 train set and 1400 test set of 2018 paper

why doesn't 2020 and 2018 papers employ SLAM algorithms directly from COLMAP and not indorectly by themsellves or are they refereing to the same slam algorithms

overfitting can be further reduced by using a cnn in the place of a gradient descent algorithm like flynn et all deep stereo 2019 i.e., essentially combining 2020 paper with this predecessor 
chaekc the first chaPpter of interduction of 2019 deep stereo for more info abou this
As a consequence, the network takes much larger strides along the direction of optimization and converges much sooner and with more accuracy than a network using standard gradient descent.

Actually, the DeepView paper has a beautiful software to customized, visualize, and render any type of MPI! It's kind of like the state of the art MPI manipulator.~\url{https://augmentedperception.github.io/deepview/}. So maybe improve the 2020 MPI html visualizer upto the standarsd of the deep view one or atleast use it to tweak and experiment the various MPI paraeters lik number of layers etc before deploying and training and testing.

Adam is better than regular stochastic gradient descent but still not superior to Flynn et al.'s~\cite{flynn_deepview_2019} implementation of learned gradient descent.